\documentclass{article}
\usepackage[utf8]{inputenc}
\usepackage{todonotes}
\usepackage{amssymb,amsmath}

\input defs.tex

\title{Computing market equilibria via convex optimization}
\author{AJ Friend \and Stephen Boyd}
\date{\today}

\begin{document}

\maketitle

\listoftodos

\section{Introduction}
we solve Fisher and Arrow-Debreu models using the formulations given in \todo{citations}. We use splitting and exploit sparsity to solve huge instances of these problems.

\subsection{Applications}
Some applications are given in Shoven and Whalley \cite{shoven1992applying}.

\section{Market definitions}
\subsection{Arrow-Debreu}

An Arrow-Debreu market has $m$ agents and $n$ goods where
agent $i$ has an initial amount $b_{ij} \in \reals_+$ of good $j$.
Agent $i$ achieves utility $u_i(x_i) \in \reals_+$ when he is is allocated a bundle of goods $x_i \in \reals^m_{+}$.
That is, he is allocated amount $x_{ij}$ of good $j$.
Generally, $u_i$ is a concave, increasing function.

Given prices $p \in \reals^n_{++}$ for each good, agent $i$ will sell his initial bundle of goods, $b_i \in \reals^n_+ $, and buy a bundle of goods $x_i$ to maximize his utility.
That is, agent $i$ solves the \emph{utility maximization problem} problem

\begin{equation}
\label{p-ump}
\begin{array}{ll}
\mbox{maximize} & u_i(x_i) \\
\mbox{subject to} & p^T x_i \leq p^T b_i \\
& x_i \geq 0.
\end{array}
\end{equation}

The \emph{Arrow-Debreu market equilibrium problem} is to find prices $p$ and an allocation matrix $x \in \reals^{m \times n}$ (where $x_i^T$ is the $i$th row of $x$) such that the following are true:
\begin{itemize}
\item $x \geq 0$
\item $\sum_{i=1}^m x_{ij} \leq \sum_{i=1}^m b_{ij}$
\item each agent achieves his optimal utility for the given prices
\end{itemize}

The second constraint simply states that the final total amount of each good cannot exceed the initial total amount.
This model is also referred to as the Walras model \cite{walras1896elements}, or the exchange model.

Arrow and Debreu showed that equilibrium prices exist in the general case of concave utility functions \cite{arrow1954existence}. However, the proof was nonconstructive and they did not provide an algorithm to compute the equilibrium.


\subsection{Fisher}

A Fisher market has $m$ agents and $n$ goods where
agent $i$ has an initial amount of money $w_i \in \reals_+$.
The total amount of good $j$ available for purchase is given by $b_j$.
Agent $i$ achieves utility $u_i(x_i) \in \reals_+$ when he is is allocated a bundle of goods $x_i \in \mathbf{R}^m_{+}$. That is, he is allocated amount $x_{ij}$ of good $j$.
Generally, $u_i$ is a concave, increasing function.

Given prices $p \in \reals^n_{++}$ for each good, agent $i$ uses his initial money $w_i$ to buy a bundle of goods $x_i$ to maximize his utility. That is, agent $i$ solves the optimization problem
\[
\begin{array}{ll}
\mbox{maximize} & u_i(x_i) \\
\mbox{subject to} & p^T x_i \leq w_i \\
& x_i \geq 0.
\end{array}
\]

The \emph{Fisher market equilibrium problem} is to find prices $p$ and an allocation matrix $x \in \reals^{m \times n}$ (where $x_i^T$ is the $i$th row of $x$) such that the following are true:
\begin{itemize}
\item $x \geq 0$
\item $\sum_{i=1}^m x_{ij} \leq b_j$
\item each agent achieves his optimal utility for the given prices
\end{itemize}

The second constraint simply states that the final total amount of each good cannot exceed the initial available amount.



\subsection{Fisher is a special case of Arrow-Debreu}
two different transformations for this
\begin{itemize}
\item add an agent and a single good (money)
\item make the initial endowment for each agent proportional,
so each agent has a constant proportion between goods, but some agents have more total goods than others. 
\end{itemize}

\section{Definitions}
We'll list some intial definitions here which will be used throughout the remainder of the paper.
For a review of convexity and convex optimization, see
\cite{boyd2009convex}.

A utility function $u: \reals^n_+ \to \reals_+$ is \emph{homothetic} if for any $\alpha > 0$ we have that $u(x) \geq u(y)$ if and only if
$u(\alpha x) \geq u(\alpha y)$.
The utility function is \emph{monotone} if $x \geq y$ implies that $u(x) \geq u(y)$.
It is \emph{homogeneous} of degree $d$ if for any $\alpha > 0$,
$u(\alpha x) = \alpha^d u(x)$.

When it doesn't cause confusion, we will write the \emph{demand relation} (or \emph{demand function}
when the relation is single-valued) of agent $i$ as $x_i(p)$, defined as the set of solutions to that agent's
utility maximization problem~(\ref{p-ump}) at given market prices $p$.

The \emph{excess demand function} is $z_i(p) = x_i(p) - b_i$.

Note that the market is in equilibrium if we find prices so that
\[
\sum_i z_i(p) \leq 0.
\]

The \emph{aggregate demand} and \emph{aggregate excess demand} functions are,
respectively,

\begin{align*}
X(p) = \sum_i x_i(p)\\
Z(p) = \sum_i z_i(p).
\end{align*}

A demand function satisfies \emph{Walras' Law} if
\[
p^T x_i(p) = p^T b,
\]
or, equivalently, $p^T z_i(p) = 0$.

A demand function satisfies the \emph{weak axiom of revealed preferences}
(WARP) if, for any two price vectors $p$ and $p'$, we have
\[
p^T x_i(p') \leq 0 \implies p'^T x_i(p) \leq 0.
\]
\todo{Note that we can interpret this condition in terms of preference cutting planes.}

If we assume Walras' Law, note that WARP is equivalent to the demand function
being a \emph{monotone operator}. That is, for any two price vectors $p$ and $p'$,
\[
\left(x(p) - x(p')\right)^T (p - p') \leq 0.
\]
\todo[inline]{Figure out maximality of this operator. How do negative prices work?
Could we just restrict the UMP to have $p^Tx \geq 0$?}

A demand function satisfies \emph{weak gross substitutability} (WGS)
if when $p$ and $p'$ are such that for some $i$, $p'_i > p_i$ and
$p'_k = p_k$ for $k \neq i$, then $x_k(p') \geq x_k(p)$ for $k \neq i$.
We can interpret this definition as requiring that if a single price increases,
then the demand for all other goods cannot decrease.

For differentiable demand functions, we could write something like
\[
\frac{\partial x_k(p)}{\partial p_i} \geq 0,\quad k \neq i.
\]

GS is a bit of a restrictive condition. WARP is a more natural condition in some
sense, as it is simply requiring that agents make consistent choices.

(Reasonable assumptions for the market are Walras' Law and 0-homogeneity.
Start with local conditions and then apply them to the aggregate functions.
Note that WARP and GS imply that demand functions supply cutting planes.)

\todo[inline]{def 17.f.2 in mas-collel. Look at page 613 for a connection
between WGS and WARP. Neither implies the other, but they both give cutting
planes. right way to phrase: GS has a `revealed preference property', pg. 614.
look at pg. 605}

existence is given by prop 17.c.2
pg. 580 gives pure exchange demand functions.
section 2.f in mas-collel for WARP
somewhere in mas-collel, he suggests and gives an example where
WARP is not additive. Weird, because monotone operators are closed under addition.
Look at the example to figure out what's up.
example 4.c.1 for failure of agg demand to be WARP
see note at end of pg 113: see EX 4.c.13 and Section 17.F



\section{Problem History}
\subsection{Formation and proof of existence}
Walras formulated his original market model in ``Elements of Pure Economics'' \cite{walras1896elements}. While he did not rigorously prove that equilibrium prices must exist, he did propose the first algorithm for computing the prices through his \emph{tatonnement} price updating scheme.

Arrow and Debreu \cite{arrow1954existence} were the first to show that equilibrium prices exist in a setting with concave utilities. The proof relies on a fixed point theorem and is thus non-constructive. As a result, it offers no obvious avenues for actually computing the equilibrium.

Arrow, Block and Hurwicz \cite{arrow1959stability} showed that for markets satisfying weak gross substitutability, a continuous-time tatonnement process would always converge to equilibrium prices.

\todo{would be nice to give the reader an idea at this point of what is tractable and what isn't. np-hard, disconnected solution sets...}

\subsection{Fisher case}
For the case of Fisher markets,
Eisenberg and Gale \cite{eisenberg1959consensus, gale1960theory} and
Eisenberg \cite{eisenberg1961aggregation} gave a convex optimization formulation for computing equilibrium first for linear utility functions, and then for the more general case of concave functions which are homogeneous with degree 1.
Jain et al. \cite{jain2005market} generalized this model to handle homothetic and quasi-concave utilities.
Examples of these utilities were introduced by Friedman
\cite{friedman1973concavity}.
\todo{rephrase}However, it is unclear if the transformation given in \cite{jain2005market} is of practical use in a convex optimization setting.

It is important to note that the Fisher model is a special case of the Arrow-Debreu model. However, the Fisher case is strictly easier than the Arrow-Debreu case. For example, Codenotti et al.\ \cite{codenotti2006leontief} show that the Arrow-Debreu problem with Leontief utilities is NP-hard, while the Fisher problem with Leontief utilities remains a convex program.

\subsection{Initial algorithms}

Scarf and others \cite{scarf2008applied,eaves1972homotopies,kuhn1968simplicial}
expanded on the fixed-point existence theorems to produce algorithms for computing equilibria based on traversing a decomposed price simplex. However, these algorithms have exponential running time in the worst case. Scarf's algorithm is also described in the text \cite{shoven1992applying}.

Smale \cite{smale1976convergent, smale1976exchange} developed Newton-based methods which are guaranteed to converge, but with no running time guarantee.

The convergent continuous-time tatonnement process of Arrow, Block and Hurwicz \cite{arrow1959stability} was extended to a discrete-time (and thus computationally implementable) version by Codenotti, McCune and Varadarajan \cite{codenotti2005marketExcess}, who were able to show that it converges for WGS markets in polynomial time.

Negishi \cite{negishi1960welfare} shows that the exchange model equilibrium is given by a convex program whose objective is a linear combination of the utility functions, but with unknown weights. Thus, the computational task is to find these positive weights. A process similar to tatonnement can be performed in the space of weights and is shown by Mantel \cite{mantel1971welfare} to be convergent under certain conditions.

\subsection{Cutting plane methods}
While working on tatonnement, Arrow, Block and Hurwicz \cite{arrow1959stability} proved a separating hyperplane lemma which would form the basis of much future work. The lemma showed that for any given (non-equilibrium) prices, a function called the aggregate excess demand provided a cutting plane in the price simplex, identifying which half-space contained the equilibrium prices.
This cutting plane result held for a class of utility functions satisfying weak gross substitutability.

An immediate corollary of this result is that for markets with the GS property, the set of equilibrium prices is convex. \todo{specifically, positive homogeneous, GS, and Walras' law} \todo{later generalized to weak GS}
However, even for some natural homogeneous (but not GS) functions, the Arrow-Debreu equilibrium may be disconnected (and thus certainly not convex) \cite{gjerstad1996multiple}. Also, \cite[p.~608]{mas1995microeconomic} shows that WGS and WARP Arrow-Debreu markets have convex equilibria.

This line of worked was continued by Polterovich, Spivak, Primak, Newman, and Nenakov \cite{nenakov1983algorithm,newman1992complexity,primak1984algorithm,primak1993converging}.

Later, a polynomial time algorithm was given by Codenotti, Pemmaraju and Varadarajan \cite{codenotti2005polynomial}. This result held for markets whose aggregate excess demand function satisfied weak gross substitutability. It combined the cutting plane result with an ellipsoid method to produce a polynomial time algorithm. This is one of the most general settings for the exchange problem and includes many utility functions. This result was later extended by McCune \cite{mccune2007extending} to markets with multi-valued aggregate excess demand functions, which includes linear utility functions.

\subsection{Convex formulations}
Even though we have a proof that Arrow-Debreu equilibria are given by a convex set, these problems still may not be easily solved with convex optimization.
The computational thread that we will follow in this paper started when Jain \cite{jain2007polynomial} rediscovered a convex formulation originally stated by Nenakov and Primak \cite{nenakov1983algorithm}. This formulation was later extended by Chen, Ye and Zhang \cite{chen2007note,chen2010equilibrium} to include some non-homogeneous utility functions. The set of functions covered by this formulation is more restrictive than the WGS setting of the previous section. \todo{how does this fit into the WGS framework?}

Again, we will mention that even for some natural homogeneous (but not GS) functions, the Arrow-Debreu equilibrium may be disconnected (and thus certainly not convex) \cite{gjerstad1996multiple}.
Also, is shown in \cite{codenotti2006leontief} that the Arrow-Debreu problem with Leontief utilities is NP-hard.
We can see that the computability of equilibrium very much depends on the properties of the utility functions.


\subsection{Other approaches}
Codenotti et al.\ \cite{codenotti2005market,codenotti2005marketCES} form a convex program for exchange economies with CES utility functions that do not satisfy WGS.

\subsection{Models with production}
Arrow-Debreu models which include the production of goods are studied in \cite{garg2014computability,jain2005market,codenotti2005marketExcess}, with an overview in \cite[Chapters~5--6]{nisan2007algorithmic}. \todo{there are many other papers to include}

\subsection{References}
Good surveys of the literature can be found in Chapters 5 and 6 of \cite{nisan2007algorithmic} and McCune's 2009 thesis \cite{mccune2009algorithmic}.

The market equilibrium problem, including existence proofs, formulations of central concepts such as demand functions, Walras' Law, and the cutting plane property, is also covered in many popular economics texts, such as \cite{varian1992microeconomic, mas1995microeconomic, luenberger1995microeconomic, kreps1990course}.

Experimental results comparing various algorithms can be found in Codenotti et al.\ \cite{codenotti2008experimental}. They compare discrete versions of tatonnement and welfare adjustment, as well as convex formulations for some specific markets.


\section{Convex optimization formulations}


\subsection{Fisher}

The Fisher equilibrium problem is solved by the following convex program when the utility functions are concave and homogeneous of degree 1 \cite{eisenberg1959consensus, gale1960theory, eisenberg1961aggregation}.
See \cite[Section~6.2]{nisan2007algorithmic} for the optimality conditions and proof:

\[
\begin{array}{ll}
\mbox{maximize} & \sum_{i=1}^m w_i \log u_i(x_i) \\
\mbox{subject to} & \sum_{i=1}^m x_{ij} \leq b_j, \quad \forall j\\
& x \geq 0.
\end{array}
\]

Here, the prices appear as dual variables.

\subsection{Arrow-Debreu}
It can be shown that a market equilibrium always exists and can be found by solving the problem

\[
\begin{array}{ll}
\mbox{find} & x, p \\
\mbox{subject to} & \nabla u_i(x_i)^T x_i \geq  \nabla_j u_i(x_i) \sum_k b_{ik} \frac{p_k}{p_j}, \forall i,j\\
& \sum_i x_{ij} \leq \sum_i b_{ij}, \forall j\\
& x_{ij} \geq 0, p_j \geq 0.
\end{array}
\]


The above problem is not convex but can be easily transformed by a simple change of variables. Specifically, let $p_j = \exp(\phi_j)$.
Then the problem becomes

\[
\begin{array}{ll}
\mbox{find} & x, \phi \\
\mbox{subject to} & \nabla u_i(x_i)^T x_i \geq  \nabla_j u_i(x_i) \sum_k b_{ik} e^{\phi_k - \phi_j}, \forall i,j\\
& \sum_i x_{ij} \leq \sum_i b_{ij}, \forall j\\
& x_{ij} \geq 0.
\end{array}
\]

This is a convex feasibility problem and can be solved in a variety of ways.
Throughout the remainder, we'll use the following formulation:

\[
\begin{array}{ll}
\mbox{find} & x, \phi \\
\mbox{subject to} & \log(\nabla u_i(x_i)^T x_i) - \log(\nabla_j u_i(x_i)) + \phi_j 
\geq \log(\sum_k b_{ik} e^{\phi_k}), \forall i,j\\
& \sum_i x_{ij} \leq \sum_i b_{ij}, \forall j\\
& x_{ij} \geq 0.
\end{array}
\]

Note that the first constraint need only be explicitly formed when $\nabla_j u_i(x_i) > 0$, and we only need to include terms in the sum in
$\log(\sum_k b_{ik} e^{\phi_k})$ when $b_{ik} > 0$. These ideas will be used later when we exploit sparsity to solve these problems efficiently.

\paragraph{Convexity}
We see that the convexity of this formulation depends on the agents' utility functions. The formulation is convex if
\[
\log(\nabla u_i(x_i)^T x_i) - \log(\nabla_j u_i(x_i))
\]
is concave.

Note that, for homogeneous functions, Euler's identity gives that
\[
\nabla u_i(x_i)^T x_i = u_i(x_i).
\]
\todo[inline]{Some special care needs to be taken here for nonsmooth functions.}

\section{Utility functions}
For each utility function, we'll list if it works in the Fisher setting, the convex formulation given by \cite{chen2010equilibrium}, or the cutting plane setting. List if homogeneous, GS or weak GS, and if demand is a function or a relation. if it satisfies walras' law, etc. Also, show how functions are representable in a DCP system.

\subsection{Linear}
\[
u(x) = a^T x
\]
\paragraph{Fisher}
This utility works in the Fisher setting, as it is homogeneous of degree 1.
\paragraph{Arrow-Debreu}
This utility works in the Arrow-Debreu setting since
\[
\log(\nabla u(x)^T x) = \log(a^T x)
\]
is concave and
\[
\log(\nabla_j u(x)) = \log(a_j)
\]
is constant, therefore convex.

\paragraph{Cutting plane}
\todo[inline]{satisfies WGS, is multi-valued, WARP}

\subsection{Constant elasticity of substitution (CES)}
CES constant elasticity of substitution have form
\[
u(x) = \left(\sum_{j=1}^n a_j x_j^\rho \right)^{1/\rho},
\]
where $-\infty < \rho \leq 1, \rho \neq 0$.
CES functions are concave for any of these values of $\rho$.
May not be tractable when $\rho < -1$.
\todo{note special cases: linear, cobbdouglas, leonteif}

\paragraph{Fisher}
This utility works in the Fisher setting, as it is concave and homogeneous of degree 1.

\paragraph{Arrow-Debreu}
Some algebra shows that 
\begin{align*}
\log(\nabla u(x)^T x) - \log(\nabla_j u(x)) =
\log\left(\sum_{j=1}^n a_j x_j^\rho \right) - \log a_j + (1-\rho) \log x_j,
\end{align*}
which is indeed concave when $0 < \rho \leq 1$.

\todo[inline]{Codenotti and McCune extend the convex formulation to handle $-1 \leq \rho < 0$.}

\paragraph{Demand}
We can derive the demand for the CES utility through the KKT conditions of the
UMP. We get that
\[
x_j = p^Tb\frac{p_j^{r-1} a_j^{1-r}}{\sum_k p_k^r a_k^{1-r}},
\]
where $r = \rho/(\rho-1)$.

\paragraph{Cutting plane} I think it is GS

\subsection{Nested CES}
see \cite{shoven1992applying} for examples and applications.  \todo{when is CES convex? nested is convex?}

\subsection{Cobb-Douglas}
\[
u(x) = \prod_{j=1}^{n} x_j^{a_j},
\]
where $\sum_j a_j = 1$.

We have that
\begin{align*}
\log(\nabla u(x)^T x) - \log(\nabla_j u(x)) =
\log x_j - \log a_j,
\end{align*}
which is indeed concave.

\paragraph{Demand}
From the optimality conditions, we find that the demand is given by
\[
x_j = \lambda \frac{a_j}{p_j},
\]
where $\lambda = \frac{p^Tb}{\sum a_j}$.

\subsection{Leontief}
\[
u(x) = \min_j a_j x_j
\]

\paragraph{Fisher}
Works for Fisher as it is homogeneous and concave.

\paragraph{Arrow-Debreu}
Does not work for Arrow-Debreu.

\subsection{Piecewise linear concave functions}
\[
u(x) = \min_k\lbrace a^{kT}x \rbrace,\quad a^k \geq 0
\]
leonteif is a special case. why is it no WGS?

\subsection{Fractional power}
 \[
 u(x) = \sum_{j=1}^n a_j (x_j+ b_j)^{d_j},
 \]
 where $a_j, b_j \geq 0$ and $0 \leq d_j \leq 1$.
 Note that these need not be homothetic. Consider $u(x,y) = \sqrt{x} + y$.

\subsection{Logarithmic}
\[
u(x) = \sum_{j=1}^n a_j \log(x_j+ b_j),
\]
where $a_j, b_j \geq 0$.  Note that these need not be homothetic.

\paragraph{Demand}
From the optimality conditions, we find that
\[
x_j = \max(0, \tau a_j/p_j - b_j),
\]
where $\tau \geq 0$ is some constant such that $\sum_j p_j x_j = p^Tb$. \todo{fix the b variable here.}
This is basically \emph{water-filling}.



\subsection{Arrow-Debreu}
things that work are:
\begin{itemize}
\item linear
\item CES with $\rho > 0$ (Codenotti may be able to write another program that handles $\rho < 0$.)
\item Cobb-Douglas
\end{itemize}
what work, what are representable
\begin{itemize}
\item complicated CES functions, nested, etc..
\item rational powers represented through SOCP
\end{itemize}

\section{Solution methods}

\subsection{Tatonnement}

\subsection{Weighted total utility}
put in the fisher stuff here. arrow-debreu can be solved this way, but we
don't know the weights. introduce re-weighting idea.

\subsection{CVXPY and SCS}

\subsection{Splitting with ADMM}
We can use an ADMM-based splitting method \cite{boyd2011distributed} to
solve the market equilibrium problem.
For this purpose, we will use an indexing notation in this section which is
more suitable for consensus ADMM.

Let $\mathcal{G}$ be an indexing set for all goods
in the market.
Agent $i$ will purchase amount $(x_i)_g$ of good $g \in \mathcal{G}$.
(We will also write $x_{ig}$ to lighten notation.)
Agent $i$'s utility is a function of the bundle $x_i \in \reals_{+}^{|G_i|}$,
where $G_i \subset \mathcal{G}$ is the set of goods involved in the utility
function $u_i$.

Note that we don't specify the actual ordering of the elements
of the vector $x_i$, as we will just care about the values associated with
each good.
When we do any operations (such as addition or scalar product) with
two vectors corresponding to the same subset of goods, we will
assume that the operation is done in a way that respects the local ordering of
the goods.

Agent $i$ is also initially endowed with a set of goods $B_i \subset \mathcal{G}$,
with allocation values $(b_i)_g = b_{ig}$, where $b_i \in \reals_{++}^{|B_i|}$.

We will treat $G$ and $B$ as \emph{relations}, \emph{correspondences}, or
\emph{multi-valued functions},
using subscript notation instead of function notation.
As we have seen, $G_i$ corresponds to the set of goods that agent $i$ is interested
in possibly purchasing.
The relation notation will also allow us to
write $G^{-1}_g$ to denote the set of agents which are interested in good
$g$.
Similarly, $B^{-1}_g$ is the set of agents initially endowed with some
positive amount of good $g$.

In the ADMM algorithm, each agent will have a local opinion for the prices
of the goods he is interested in purchasing or selling.
Let his local prices
be $\phi_i \in \reals_+^{|G_i \cup B_i|}$.

Note that
\[
\bigcup_{i=1}^n G_i = \bigcup_{i=1}^n B_i = \mathcal{G}.
\]

For any variable $z \in \reals^{|\mathcal{G}|}$ which contains values for
each good in the market, we may write $z_{G_i}$ to represent the subvector
of $z$ whose elements correspond to the goods in $G_i$.

For agent $i$, let $f_i(x_i, \phi_i)$ be the indicator function for the constraints
\[
\begin{array}{c}
\log(\nabla u_i(x_i)^T x_i) - \log(\nabla_g u_i(x_i)) + \phi_{ig} \geq  \log\left(\sum\limits_{k \in B_i} b_{ik} e^{\phi_{ik}}\right),\quad \forall g \in G_i\\
x_{ig} \geq 0, \quad \forall g \in G_i.
\end{array}
\]

\paragraph{ADMM problem form} Then the problem we'd like to solve is given by

\[
\begin{array}{ll}
\mbox{minimize} & \sum_i f_i(x_i, \phi_i) \\
\mbox{subject to} & \sum\limits_{i \in G^{-1}_g} x_{ig} = \sum\limits_{i \in B^{-1}_g} b_{ig},\quad \forall g \in \mathcal{G}\\
& \phi_{ig} = \Phi_g,\quad \forall i \in G^{-1}_g \cup B^{-1}_g.
\end{array}
\]

The first set of constraints are the global resource constraints; the amount of
goods purchased by agents cannot exceed the total initial endowed amount.
We can assume equality here (instead of an upper bound)
since constraints are tight at the solution. This equality formulation gives
a simpler ADMM iteration.

The second set of constraints are consensus constraints on the price variables,
$\phi_i$. It simply says that all agents with an opinion, $\phi_{ij}$, on the price
of good $g$ must agree. They agree through a global consensus variable $\Phi$.


\paragraph{ADMM algorithm}
Note that we have a generalized consensus problem in the price variables,
$\phi$, and a generalized sharing problem in the allocation variables, $x$.

The resulting ADMM algorithm is 
\begin{align}
\label{a-xtild}
\tilde{x}^k_g &:= \frac{1}{|G^{-1}_g|} \left( \sum_{i \in G^{-1}_g} x^k_{ig} - \sum_{i \in B^{-1}_g} b_{ig}\right)\\
\label{a-phibar}
\bar{\phi}^k_g &:= \frac{1}{ |G^{-1}_g \cup B^{-1}_g| } \sum_{i \in G^{-1}_g \cup B^{-1}_g}\phi^k_{ig}\\
u^{k+1} &:= u^k + \tilde{x}^k\\
w_i^{k+1} &:= w_i^k + \phi^k_i - \bar{\phi}^k_{G_i}\\
\label{a-prox}
x_i^{k+1}, \phi_i^{k+1} &:= \mbox{prox}_{f_i}(x_i^k - \tilde{x}^k_{G_i} - u^{k+1}_{G_i},
\bar{\phi}^k_{G_i} - w_i^{k+1}).
\end{align}

This algorithm needs to be explained.
In equation~(\ref{a-xtild}), $\tilde{x}^k_g$ gives a measure of the violation of
the global resource constraint, normalized by the number of agents participating
in considering that good.
In equation~(\ref{a-phibar}), $\bar{\phi}^k_g$ gives the average of the price
opinions $\phi^k_g$ over all agents with an interest in the price of good $g$.
Dual variables $u^k$ and $w^k_i$ are updated in the next two equations.
Note that $\tilde{x}^k$, $\bar{\phi}^k$, and $u^k$ are \emph{global} variables
and that $w_i^k$, $x_i^{k+1}$, and $\phi_i^{k+1}$ are \emph{local} variables
for each agent. In equation~(\ref{a-prox}), each agent compute the projection
onto his local optimality constraints. Each agent emits their local results and the
global computation of $\tilde{x}^{k+1}_g$ and $\bar{\phi}^{k+1}_g$ continues in the next
step.

The algorithm above is decentralized and parallelizable.
Each agent can compute their prox operator separately and in parallel.
The results of these prox operators are aggregated, averaged, and then distributed
back to the individual agents for the next iteration.

\subsection{Monotone operators}
For many utility functions \todo{when?}, the excess demand function (or correspondence,
or relation) of an agent is a
monotone operator.
That is,
\begin{equation}
\label{e-monotone}
\left[z(p) - z(q) \right]^T \left[p - q\right] \leq 0.
\end{equation}

If we assume that Walras' Law holds, \ie, each agent spends their entire
budget, or $p^T z(p) = p^T x(p) - p^T b = 0$, then the operator monotonicity has
a nice economic interpretation in terms of \emph{preferences}.

Expanding inequality~(\ref{e-monotone}) and applying Walras' Law, we find that
\[
p^T z(q) + q^T z(p) \geq 0.
\]

We thus see that the implication
\begin{equation}
\label{e-preference}
p^T z(q) \leq 0 \implies q^T z(p) \geq 0
\end{equation}
is equivalent \todo{or just implies? is it actually true that they are
equivalent?} to inequality~(\ref{e-monotone}) whenever Walras' Law holds.

We interpret (\ref{e-preference}) as revealing that the agent prefers prices
$p$ over prices $q$. If $p^T z(q) \leq 0$, then $p^T x(q) \leq p^T b$, implying
that both bundles $x(p)$ and $x(q)$ are affordable (feasible) at prices $p$.
Since $x(p)$ is an optimal bundle at prices $p$, we know that it must be at least
as preferred at $x(q)$, or that $u(x(p)) \geq u(x(q))$.

Inequality~(\ref{e-monotone}) already shows that implication~(\ref{e-preference})
holds, but to interpret the result economically, suppose that the
implication did not hold, \ie, that $q^T z(p) < 0$. We would then have that
$q^T x(p) < q^T b$, which implies that $x(p)$ and $x(q)$ are both affordable
bundles at at prices $q$. However, we know that $u(x(p)) \geq u(x(q))$, so
bundle $x(p)$ must be optimal, and Walras' Law implies $q^T x(p) = q^T b$, which
gives a contradiction. 

In the case of strict inequality, if $p^T z(q) < 0$, then $q^T z(p) > 0$.
That is, $x(p)$ is revealed as preferred to $x(q)$, and $x(p)$ is not within
budget at prices $q$, so prices $p$ are preferred by the agent.

Note that if each agent's excess demand function is a monotone operator, then
the market's aggregate excess demand function is also monotone. If these
operators are \emph{maximal} monotone, then we can use the this framework for
solving these problems. \todo[inline]{However, it is unclear if and when these
operators are maximal.}

Even in the case that the operators are monotone but not maximal, we still get
a cutting-plane by computing the demand function. That is, for any prices $p$
and optimal prices $p^*$, we have that
\[
p^{*T} z(p) \geq 0,
\]
which gives us a cutting-plane in the price simplex.
A cutting-plane algorithm can then be used to compute the optimal prices

\paragraph{Computing the resolvent}
The resolvent of the excess demand operator is
\[
(I + \lambda z)^{-1}(p) = q,
\]
which we parse to interpret as
\[
p = q + \lambda x(q).
\]
That is, given prices $p$, we must find prices $q$ such that the agent's UMP
at prices $q$ satisfy the above quality.

XXX: We can write down the optimality conditions, but these may put the same
constraints on the utility functions as the Ye formulation. We can always
substitute $p = q + \lambda x(q)$ into the opt conditions to remove either
$q$ or $x$. But we end up with a non-convex problem. We optimize over the a
non-convex region, something like the space outside of a unit ball.
We may also be able to formulate the problem as being bi-convex in $x$ and $q$,
but I'm not sure how helpful that would be.

\subsection{Cutting plane}
\paragraph{Approximations?}
can i approximate the exponential cone to do an approximate prox evaluation
at each step, allowing for SOCP solvers that can't handle exponential cones?

what if we approximate everything with a quadratic around the current iterate for an approximate projection? subset or superset of actual agent cutting set? does this simplification make updates with an analytic form, allowing for very fast iteration?


\section{Computational examples}
\missingfigure{Add a yet to be made figure here.}

\section{Application examples}
\begin{itemize}
\item traffic
\item routing
\item spectrum management
\item others?
\end{itemize}

\newpage
\bibliographystyle{alpha}
\bibliography{bibliography.bib}

\end{document}


