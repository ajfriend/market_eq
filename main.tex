\documentclass[12pt]{article}
\usepackage[utf8]{inputenc}
\usepackage{todonotes}
\usepackage{amssymb,amsmath}
\usepackage[small,bf]{caption}

% add bib to ToC
\usepackage{tocbibind}

\input defs.tex

\title{Computing market equilibria via convex optimization}
\author{A.J. Friend \and Stephen Boyd}
\date{\today}

\begin{document}

\maketitle

\begin{abstract}
We consider the pure exchange market equilibrium problem of finding prices
such that the market clears, \ie, demand does not exceed supply.
This is a classic problem in economics, and has been discussed recently
in the literature from a computational complexity perspective.
Indeed, depending on the utility functions of the agents in the market,
finding equilibrium prices can be NP-hard.
This paper will focus on the numerical computation
of equilibria for a subset of markets whose equilibrium
problem can be cast as a convex program.
We introduce algorithms which scale to huge markets and can be computed
in a distributed fashion.
We report numerical results for large random markets and for
application examples in economic policy, wireless spectrum management,
and Internet advertising auctions.
\end{abstract}

\newpage
\tableofcontents
%\newpage
%\listoftodos
\newpage

\section{Introduction}
We will consider exchange market equilibrium problems first introduced by
Walras in his in ``Elements of Pure Economics''
\cite{walras1896elements}.
Walras considers a market with agents trading
goods at prescribed prices to maximize their own utility functions.
The market equilibrium problem is that of finding prices for the goods
such that the total demand of the agents does not exceed the total amount
of each good supplied by the agents for trade.

In full generality, Walras' model includes \emph{production}, \ie, processes by which existing goods can be consumed to produce new goods which may enter the market.
We restrict this paper to the pure exchange market, which allows for trading of existing goods, but not production.

Walras suggested his natural \emph{tatonnment} price adjustment scheme as a means of computing equilibrium prices.
However, the convergence of tatonnement and conditions for existence of equilibrium prices remained an open problems for many years.

The Nobel Laureates Arrow and Debreu were the first to show that equilibrium
prices exist under mild conditions on the utility functions of agents\cite{arrow1954existence}.
However, these proofs relied on fixed-point theorems and were non-constructive.
As such, they offered no practical means by which to compute these equilibria.

This paper will investigate convex formulations of the market
equilibrium problem which were rediscovered by Jain \cite{jain2007polynomial}.
The original formulations for linear utility functions
%todo check
were stated by Nenakov and Primak \cite{nenakov1983algorithm}.
Jain's convex formulation was later extended by Chen, Ye, and Zhang \cite{chen2007note, chen2010equilibrium} to include some non-homogeneous
utility functions.

Eisenberg and Gale \cite{eisenberg1959consensus, gale1960theory, eisenberg1961aggregation} gave a convex program for a special case of Walras' model which was independently developed by Fisher. %todo citation?
This restricted case allows us to handle a larger family of utility functions
with the convex framework.

Our approach in this paper is to use the \emph{alternating direction method of multipliers} (ADMM) \cite{boyd2011distributed} to produce a distributed and scalable algorithm for solving convex formulations of the market equilibrium problem.

We provide examples and convergence results for large random markets, and 
for application examples in economic policy, wireless spectrum management,
and Internet advertising auctions.

\section{Market definitions}
In this section, we give exact mathematical definitions for the market equilibrium problems that we will be studying. We define the exchange model,
which deals with agents trading goods, and a special case called the Fisher model. In the Fisher model, agents do not start with an initial endowment of
goods and do not trade for goods. Instead they purchase goods with some
initial wealth from a store of globally available goods.

\subsection{Exchange model}
An \emph{exchange market} has $m$ agents and $n$ goods where
agent $i$ has an initial endowment of goods $b_i \in \reals_+^n$.
Agent $i$ achieves utility $u_i(x_i) \in \reals_+$ when he is is allocated a
bundle of goods $x_i \in \reals^n_{+}$,
that is, he is allocated amount $x_{ij}$ of good $j$.
In this paper, $u_i$ will always be concave and increasing.

Given prices $p \in \reals^n_{++}$ for the goods, agent $i$ will sell his
initial bundle of goods, $b_i$, and buy a bundle of goods
$x_i$ to maximize his utility.
That is, agent $i$ solves the \emph{utility maximization problem} (UMP)

\begin{equation}
\label{p-ump}
\begin{array}{ll}
\mbox{maximize} & u_i(x_i) \\
\mbox{subject to} & p^T x_i \leq p^T b_i \\
& x_i \geq 0,
\end{array}
\end{equation}
with optimization variable $x_i$.

When it won't be confused with the optimization variable,
we will denote the \emph{set} of solutions to the UMP as $x^\star_i(p)$,
and refer to it as the \emph{demand} of agent $i$ at prices $p$.
Formally, $x_i^\star : \reals^n_{++} \to 2^{\reals^n_+}$ is a
\emph{relation}, \emph{set-valued mapping}, or \emph{correspondence}.
When the optimal bundle is unique, we may think of
the demand as a \emph{function}.
%todo reasonable notation?
XXX: reasonable notation for relations?

The \emph{exchange market equilibrium problem} is to find prices $p$
such that the demand for goods in the market does not exceed the supply
provided by the initial agent endowments.
When the demands are all functions, we can write this constraint as
\[
\sum_{i=1}^m x^\star_{i}(p) \leq \sum_{i=1}^m b_i = B,
\]
where $B \in \reals^n_{++}$ is the
the total amount of goods in the market available for purchase.

For the markets we will consider, it will be the case that demand is
always exactly equal to supply at optimal prices,
so we may write the inequality as an equality when it is convenient. 

In the case of demand relations, we can write the equilibrium
condition as
\[
0 \in \sum_{i=1}^m x^\star_{i}(p) - B.
\]
This is equivalent to saying that there exist some
allocations $y_i \in x^\star_i(p)$ such that
\[
\sum_{i=1}^m y_i = B.
\]

The exchange model is also referred to as the Walras model \cite{walras1896elements},
the pure exchange model, or the Arrow-Debreu model.

\subsection{Fisher model}
A Fisher market has $m$ agents and $n$ goods where
agent $i$ has an initial amount of money, or wealth, $w_i \in \reals_{++}$.
The total amount of good $j$ available for purchase in the market is given by
$B_j \in \reals_{++}$.
Agent $i$ achieves utility $u_i(x_i) \in \reals_+$ when he is is allocated a
bundle of goods $x_i \in \mathbf{R}^m_{+}$. That is, he is allocated amount
$x_{ij}$ of good $j$.
Again, we will only consider functions $u_i$ which are concave and increasing.

Given prices $p \in \reals^n_{++}$ for each good, agent $i$ uses his initial
wealth $w_i$ to buy a bundle of goods $x_i$ to maximize his utility.
That is, agent $i$ solves the \emph{Fisher utility maximization problem}
\begin{equation}
\label{p-fisher-ump}
\begin{array}{ll}
\mbox{maximize} & u_i(x_i) \\
\mbox{subject to} & p^T x_i \leq w_i \\
& x_i \geq 0.
\end{array}
\end{equation}
We will again use $x^\star_i(p)$ to denote the demand relation (or function)
for agent $i$, \ie, the set of solutions to the Fisher UMP~(\ref{p-fisher-ump}).

The \emph{Fisher market equilibrium problem} is to find prices $p$ such that
\[
0 \in \sum_{i=1}^m x^\star_{i}(p) - B.
\]
Again, this is just the requirement that demand equals supply
at prices $p$.

\paragraph{Fisher is a special case of Arrow-Debreu}
We can cast any Fisher equilibrium problem as an exchange market problem using the following transformation.

Let $w_i$ be the wealth of agent $i$ and $B$ be the total amount of goods in
the Fisher system.
Let $W = \sum_{i=1}^m w_i$ be the total wealth in the Fisher market.
To form the exchange market corresponding to the Fisher system, assign an initial bundle of goods
$b_i = B w_i/W$ to agent $i$.
Since the scaling of $p$ in the exchange problem is arbitrary,
this gives the correct proportion of wealth to each agent with the correct
total amount of goods in the market.

Using this transformation, we see that any Fisher problem can be cast as an
exchange problem. Thus, any convex formulation for exchange problems can be
used to solve Fisher markets. Later, we will see a convex formulation
which only applies to Fisher markets. This will allow us to solve
Fisher markets for a larger class of utility functions than work for
exchange models.

\section{Properties of markets and utility functions}
We'll list some initial definitions here which will be used throughout the remainder of the paper.
These definitions will allow us to discuss various aspects of the
equilibrium problem, such as describing the set of equilibrium prices
in terms of whether such prices exist, if they are unique, and if the set is convex or if there are multiple, disconnected equilibria.
These properties will allow allow us to describe when market problems
are computationally tractable, and when markets fit into
various convex formulations.
For a review of convexity and convex optimization, see
\cite{boyd2009convex}.


\subsection{Utility functions}
A utility function $u: \reals^n_+ \to \reals_+$ is \emph{homothetic} if for any $\alpha > 0$ we have that $u(x) \geq u(y)$ if and only if
$u(\alpha x) \geq u(\alpha y)$.
The utility function is \emph{monotone} if $x \geq y$ implies that $u(x) \geq u(y)$.
It is \emph{homogeneous} of degree $d$ if for any $\alpha > 0$,
$u(\alpha x) = \alpha^d u(x)$.

For homogeneous functions of degree 1, Euler's identity gives us that
\[
\nabla u_i(x_i)^T x_i = u_i(x_i).
\]
%todo special care for nonsmooth functions?


\subsection{Demand and excess demand}
As we saw previously, we will write the \emph{demand relation}
(or \emph{demand function}
when the relation is single-valued) of agent $i$ as $x^\star_i(p)$,
defined as the set of solutions to that agent's
utility maximization problem~(\ref{p-ump}) at given market prices $p$.
We define the \emph{excess demand relation} as 
\[
z_i(p) = x^\star_i(p) - b_i.
\]

Note that the market is in equilibrium if we find prices so that
\[
0 \in \sum_{i=1}^m z_i(p).
\]

The \emph{aggregate demand} and \emph{aggregate excess demand} relations are,
respectively,

\begin{align*}
X(p) &= \sum_{i=1}^m x^\star_i(p),\\
Z(p) &= \sum_{i=1}^m z_i(p).
\end{align*}

We abuse notation a bit when we say that n agent's demand relation $x(p)$ satisfies \emph{Walras' Law} if
\[
p^T x^\star(p) = p^T b,
\]
or, equivalently, $p^T z(p) = 0$.
Formally, we mean that $p^T x = p^T b$ for all $x \in x^\star(p)$.
Walras' Law simply requires that an agent spends his entire budget to
maximize his utility. When agent utilities are concave and increasing,
this is a natural consequence.

\subsection{The weak axiom of revealed preferences}
A demand satisfies the \emph{weak axiom of revealed preferences}
(WARP) if, for any two price vectors $p$ and $q$, we have the equivalent
conditions
\begin{align*}
p^T x(q) \leq p^T b &\implies q^T x(p) \geq q^T b,\\
p^T z(q) \leq 0 &\implies q^T z(p) \geq 0.
\end{align*}

We can interpret WARP as follows.
If bundle $x(p)$ is revealed to be at least as preferred as bundle
$x(q)$ (because they are both within budget at prices $p$), then
$x(p)$ must be at least as expensive as bundle $x(q)$ at prices $q$.
WARP can be interpreted as requiring that agent's preferences be
consistent in some sense.

\subsection{Gross substitutability}
A demand $x^\star(p)$ satisfies \emph{weak gross substitutability} (WGS)
if when $p$ and $q$ are such that for some $i$, $q_i > p_i$ and
$q_k = p_k$ for $k \neq i$, then $x_k(q) \geq x_k(p)$ for $k \neq i$.
We can interpret this definition as requiring that if a single price increases,
then the demand for all other goods cannot decrease.

For differentiable demand functions, we could write something like
\[
\frac{\partial x_k(p)}{\partial p_i} \geq 0,\quad k \neq i.
\]


\subsection{Cutting plane implications}
WARP and WGS are significant because they both imply that
the aggregate excess demand functions supply cutting planes which can
be used to find optimal market prices.
That is, for any prices $p$ and optimal prices $p^\star$, we have that
\[
p^{\star T} Z(p) \geq 0.
\]
Thus, for any market satisfying WGS or WARP, the set of optimal prices
is convex.
%todo this is easy to see for WARP. what's the proof for WGS?

XXX: GS is a bit of a restrictive condition. WARP is a more natural condition in some
sense, as it is simply requiring that agents make consistent choices.

(Reasonable assumptions for the market are Walras' Law and 0-homogeneity.
Start with local conditions and then apply them to the aggregate functions.
Note that WARP and GS imply that demand functions supply cutting planes.)

XXX: def 17.f.2 in mas-collel. Look at page 613 for a connection
between WGS and WARP. Neither implies the other, but they both give cutting
planes. right way to phrase: GS has a `revealed preference property', pg. 614.
look at pg. 605

existence is given by prop 17.c.2
pg. 580 gives pure exchange demand functions.
section 2.f in mas-collel for WARP
somewhere in mas-collel, he suggests and gives an example where
WARP is not additive. Weird, because monotone operators are closed under addition.
Look at the example to figure out what's up.
example 4.c.1 for failure of agg demand to be WARP
see note at end of pg 113: see EX 4.c.13 and Section 17.F

\subsection{Monotone operators}
For many utility functions, the associated demand is
a monotone operator.


\section{Problem History}
\subsection{Formation and proof of existence}
Walras formulated his original market model in ``Elements of Pure Economics''
\cite{walras1896elements}. While he did not rigorously prove that equilibrium
prices must exist, he did propose the first algorithm for computing the prices
through his \emph{tatonnement} price updating scheme.

Arrow and Debreu \cite{arrow1954existence} were the first to show that
equilibrium prices exist in a setting with concave utilities.
The proof relies on a fixed point theorem and is thus non-constructive.
As a result, it offers no obvious avenues for actually computing the
equilibrium.

Arrow, Block and Hurwicz \cite{arrow1959stability} showed that for markets
satisfying weak gross substitutability, a continuous-time tatonnement process
would always converge to equilibrium prices.

%todo would be nice to give the reader an idea at this point of what is
%tractable and what isn't. np-hard, disconnected solution sets...

\subsection{Fisher case}
For the case of Fisher markets,
Eisenberg and Gale \cite{eisenberg1959consensus, gale1960theory} and
Eisenberg \cite{eisenberg1961aggregation} gave a convex optimization
formulation for computing equilibrium first for linear utility functions,
and then for the more general case of concave functions which are homogeneous
with degree 1.
Jain et al. \cite{jain2005market} generalized this model to handle homothetic
and quasi-concave utilities.
Examples of these utilities were introduced by Friedman
\cite{friedman1973concavity}.
%todo rephrase}
However, it is unclear if the transformation given in
\cite{jain2005market} is of practical use in a convex optimization setting.

It is important to note that the Fisher model is a special case of the
Arrow-Debreu model. However, the Fisher case is strictly easier
%todo easier in what sense? consider \cite{chen2009spending}
than the
Arrow-Debreu case. For example, Codenotti et al.\ \cite{codenotti2006leontief}
show that the Arrow-Debreu problem with Leontief utilities is NP-hard, while
the Fisher problem with Leontief utilities remains a convex program.

\subsection{Initial algorithms}

Scarf and others \cite{scarf2008applied,eaves1972homotopies,kuhn1968simplicial}
expanded on the fixed-point existence theorems to produce algorithms for
computing equilibria based on traversing a decomposed price simplex.
However, these algorithms have exponential running time in the worst case.
Scarf's algorithm is also described in the text \cite{shoven1992applying}.

Smale \cite{smale1976convergent, smale1976exchange} developed Newton-based
methods which are guaranteed to converge, but with no running time guarantee.

The convergent continuous-time tatonnement process of Arrow, Block and Hurwicz
\cite{arrow1959stability} was extended to a discrete-time
(and thus computationally implementable) version by Codenotti, McCune and
Varadarajan \cite{codenotti2005marketExcess}, who were able to show that it
converges for WGS markets in polynomial time.

Negishi \cite{negishi1960welfare} shows that the exchange model equilibrium is
given by a convex program whose objective is a linear combination of the
utility functions, but with unknown weights.
Thus, the computational task is to find these positive weights.
A process similar to tatonnement can be performed in the space of weights
and is shown by Mantel \cite{mantel1971welfare} to be convergent under certain
conditions.

\subsection{Cutting plane methods}
While working on tatonnement, Arrow, Block and Hurwicz \cite{arrow1959stability}
proved a separating hyperplane lemma which would form the basis of much future
work.
The lemma showed that for any given (non-equilibrium) prices, a function called
the aggregate excess demand provided a cutting plane in the price simplex,
identifying which half-space contained the equilibrium prices.
This cutting plane result held for a class of utility functions satisfying
weak gross substitutability.

An immediate corollary of this result is that for markets with the GS property,
the set of equilibrium prices is convex.
%todo specifically, positive homogeneous, GS, and Walras' law. later generalized to weak GS
However, even for some natural homogeneous (but not GS) functions,
the Arrow-Debreu equilibrium may be disconnected
(and thus certainly not convex) \cite{gjerstad1996multiple}.
Also, \cite[p.~608]{mas1995microeconomic} shows that WGS and WARP Arrow-Debreu
markets have convex equilibria.

This line of worked was continued by Polterovich, Spivak, Primak, Newman,
and Nenakov \cite{nenakov1983algorithm,newman1992complexity,primak1984algorithm,primak1993converging}.

Later, a polynomial time algorithm was given by Codenotti, Pemmaraju and
Varadarajan \cite{codenotti2005polynomial}.
This result held for markets whose aggregate excess demand function satisfied
weak gross substitutability.
It combined the cutting plane result with an ellipsoid method to produce a
polynomial time algorithm.
This is one of the most general settings for the exchange problem and includes
many utility functions. This result was later extended by McCune
\cite{mccune2007extending} to markets with multi-valued aggregate excess
demand functions, which includes linear utility functions.

\subsection{Convex formulations}
Even though we have a proof that Arrow-Debreu equilibria are given by a
convex set, these problems still may not be easily solved with convex
optimization.
The computational thread that we will follow in this paper started when
Jain \cite{jain2007polynomial} rediscovered a convex formulation originally
stated by Nenakov and Primak \cite{nenakov1983algorithm}.
This formulation was later extended by Chen, Ye and
Zhang \cite{chen2007note,chen2010equilibrium} to include some non-homogeneous
utility functions.
The set of functions covered by this formulation is more restrictive than the
WGS setting of the previous section.
%todo how does this fit into the WGS framework?

Again, we will mention that even for some natural homogeneous (but not GS)
functions, the Arrow-Debreu equilibrium may be disconnected
(and thus certainly not convex) \cite{gjerstad1996multiple}.
Also, is shown in \cite{codenotti2006leontief} that the Arrow-Debreu
problem with Leontief utilities is NP-hard.
We can see that the computability of equilibrium very much depends on the
properties of the utility functions.


\subsection{Other approaches}
Codenotti et al.\ cite{codenotti2005market, codenotti2005marketCES} form a
convex program for exchange economies with CES utility functions that do not
satisfy WGS.

\subsection{Models with production}
Arrow-Debreu models which include the production of goods are studied
in \cite{garg2014computability,jain2005market,codenotti2005marketExcess},
with an overview in \cite[Chapters~5--6]{nisan2007algorithmic}.
%todo there are many other papers to include

\subsection{References}
Good surveys of the literature can be found in Chapters 5 and 6 of
\cite{nisan2007algorithmic} and McCune's 2009 thesis \cite{mccune2009algorithmic}.

The market equilibrium problem, including existence proofs, formulations of
central concepts such as demand functions, Walras' Law, and the cutting plane
property, is also covered in many popular economics texts, such as
\cite{varian1992microeconomic, mas1995microeconomic,
luenberger1995microeconomic, kreps1990course}.

Experimental results comparing various algorithms can be found in Codenotti
et al.\ \cite{codenotti2008experimental}.
They compare discrete versions of tatonnement and welfare adjustment,
as well as convex formulations for some specific markets.


\section{Convex optimization formulations}
In this section, we cover the convex formulations which can be used to solve
the Fisher and exchange market equilibrium problems.

\subsection{Exchange}
Following the formulations given in \cite{jain2007polynomial, chen2007note, nenakov1983algorithm}, it can be shown that the optimality conditions for 
the exchange market are equivalent to the conditions
\[
\begin{array}{ll}
& \nabla u_i(x_i)^T x_i \geq  \nabla_j u_i(x_i) \sum_k b_{ik} \frac{p_k}{p_j}, \quad \forall i,j\\
& \sum_i x_{ij} \leq \sum_i b_{ij},\quad \forall j\\
& x_{ij} \geq 0, p_j \geq 0.
\end{array}
\]
For a proof of this equivalence, see Appendix~\ref{sec:fisher_proof}.

To obtain a convex program, we first apply some transformations to the
problem.
Specifically, let $p_j = \exp(\phi_j)$, and take the logarithm of
the first set of constraints.

We obtain the optimization problem %todo fix typesetting
\begin{equation}
\label{p-exchange}
\begin{array}{ll}
\mbox{find} & x, \phi \\
\mbox{subject to} & \log(\nabla u_i(x_i)^T x_i) - \log(\nabla_j u_i(x_i)) + \phi_j 
\geq \log(\sum_k b_{ik} e^{\phi_k}),\quad \forall i,j\\
& \sum_i x_{ij} \leq \sum_i b_{ij},\quad \forall j\\
& x_{ij} \geq 0.
\end{array}
\end{equation}

Note that the optimization problem is convex if all the utility functions
in the market are such that the term
\[
\log(\nabla u_i(x_i)^T x_i) - \log(\nabla_j u_i(x_i))
\]
is concave for all $i$ and $j$.

Note that the first constraint need only be explicitly formed when
$\nabla_j u_i(x_i) > 0$, and we only need to include terms in the sum in
$\log(\sum_k b_{ik} e^{\phi_k})$ when $b_{ik} > 0$.
These ideas will be used later when we exploit sparsity to solve these
problems efficiently.

\subsection{Fisher}

When the utility functions are concave and homogeneous of degree 1,
the Fisher equilibrium problem is solved by the convex program
\begin{equation}
\label{p-fisher}
\begin{array}{ll}
\mbox{maximize} & \sum_{i=1}^m w_i \log u_i(x_i) \\
\mbox{subject to} & \sum_{i=1}^m x_i \leq B\\
& x \geq 0,
\end{array}
\end{equation}
originally given by \cite{eisenberg1959consensus, gale1960theory, eisenberg1961aggregation}.
For a proof, see Appendix~\ref{sec:fisher_proof} or \cite[\S~6.2]{nisan2007algorithmic}.

The objective in this convex program is sometimes referred to as a \emph{weighted aggregate utility}, where the weights are given by the amount
of wealth $w_i$ possessed by agent $i$.
Note that only allocations $x_i$ appear as variables in the optimization
problem.
The equilibrium prices can be recovered as dual variables.

\section{Utility functions}
In this section, we'll describe various families of utility functions
and state whether they fit into the convex programming framework
for Fisher and exchange markets. We will also note if they
satisfy GS or WARP, in which case the aggregate excess demand function
provides a cutting plane for the optimal prices. It will also
be valuable to note how the convex program expressions associated
with the utility functions may be expressed in a disciplined
convex programming framework \cite{grant2006disciplined}.

%todo provide some references for when these utilities come up in econ

\subsection{Linear}
Linear utility functions have the form
\[
u(x) = a^T x.
\]

\paragraph{Fisher}
This utility is compatible with the Fisher problem~(\ref{p-fisher}), as it is homogeneous of degree 1.

\paragraph{Arrow-Debreu}
Th utility is also compatible with the exchange problem~(\ref{p-exchange}),
since
\[
\log(\nabla u(x)^T x) - \log(\nabla_j u(x))  = \log(a^T x) - \log(a_j)
\]
is concave.

\paragraph{Demand}
Generally, the demand for this utility is a relation as the solution
to the UMP is often not unique.

\paragraph{Cutting plane}
satisfies WGS, WARP, demand is a relation, demand is a monotone operator...

\subsection{Constant elasticity of substitution (CES)}
Constant elasticity of substitution (CES) functions have the form
\[
u(x) = \left(\sum_{j=1}^n a_j x_j^\rho \right)^{1/\rho}.
\]
For $-\infty < \rho \leq 1, \rho \neq 0$, the demand is a concave and
increasing function.

It is useful to note that when $\rho = 1$, we have the linear utility function. As $\rho$ approaches $0$ and $-\infty$ we recover the Cobb-Douglas and Leonteif
utility functions in the limit. %todo check which one is which

\paragraph{Fisher}
This utility is compatible with Fisher convex formulation, as it is concave and homogeneous of degree 1.

\paragraph{Exchange}
Some algebra shows that 
\begin{align*}
\log(\nabla u(x)^T x) - \log(\nabla_j u(x)) =
\log\left(\sum_{k=1}^n a_k x_k^\rho \right) - \log a_j + (1-\rho) \log x_j,
\end{align*}
which is only concave when $0 < \rho \leq 1$.

Codenotti and McCune \cite{codenotti2005marketCES} were able to provide a convex formulation to include
CES functions with $-1 \leq \rho < 0$.

\paragraph{Demand}
The demand for the CES utility is unique, and can therefore be thought
of as a function.
We can derive the demand through the KKT conditions of the
UMP. We get that
\[
x^\star_j(p) = p^Tb\frac{p_j^{r-1} a_j^{1-r}}{\sum_k p_k^r a_k^{1-r}},
\]
where $r = \rho/(\rho-1)$.

\paragraph{Cutting plane} I think it is GS, WARP? monotone?

\subsection{Cobb-Douglas}
The Cobb-Douglas utility has the form
\[
u(x) = \prod_{j=1}^{n} x_j^{a_j},
\]
where $\sum_j a_j = 1$.

\paragraph{Fisher}
The utility is concave and homogeneous, so it is compatible with the Fisher
convex formulation.
\paragraph{Exchange}
We have that
\begin{align*}
\log(\nabla u(x)^T x) - \log(\nabla_j u(x)) =
\log x_j - \log a_j,
\end{align*}
which is indeed concave, so the utility is also compatible with
the exchange convex formulation.

\paragraph{Demand}
From the optimality conditions, we find that the demand is given by
\[
x^\star_j(p) = \frac{a_j p^Tb}{p_j\sum_k a_k}.
\]

\subsection{Leontief}
The Leontief utility has the form
\[
u(x) = \min_j a_j x_j.
\]

\paragraph{Fisher}
Works for Fisher as it is homogeneous and concave.

\paragraph{Arrow-Debreu}
Does not work for Arrow-Debreu.

\paragraph{Demand}
probably a relation

\paragraph{Cutting plane}
Leonteif allows for multiple disconnected equilibria in the exchange problem,
so I don' think the demand function can be monotone. what about
GS or WARP? where does it break down?

\subsection{Piecewise linear concave functions}
\[
u(x) = \min_k\lbrace a^{kT}x \rbrace,\quad a^k \geq 0
\]
leonteif is a special case. why is it not WGS?

\subsection{Fractional power}
\[
u(x) = \sum_{j=1}^n a_j (x_j+ c_j)^{d_j},
\]
where $a_j, c_j \geq 0$ and $0 \leq d_j \leq 1$.
Note that these need not be homothetic. Consider $u(x,y) = \sqrt{x} + y$.

\paragraph{Fisher}
Note that these functions are \emph{not} homogeneous of degree 1, and thus
are not compatible with the Fisher program~(\ref{p-fisher}).

\paragraph{Exchange}
However, we find that 
\begin{align*}
\log(\nabla u(x)^T x) - \log(\nabla_j u(x))
&= \log\left(\sum_{k=1}^n a_k d_k (x_k+c_k)^{d_k} - \frac{a_k d_k c_k}{(x_k + c_k)^{1-d_k}} \right)\\
&\quad- \log(a_j d_j) + (1-d_j)\log (x_j + c_j),
\end{align*}
which is seen to be concave, so this utility is compatible with
the exchange program~(\ref{p-exchange}).
Fisher problems with this utility can be solved by converting the
problem to the exchange form.

\paragraph{Demand}
should be able to figure out.

\paragraph{Cutting plane}
WGS, WARP, monotone?


\subsection{Logarithmic}
\[
u(x) = \sum_{j=1}^n a_j \log(x_j+ c_j),
\]
where $a_j, c_j \geq 0$.
Note that these utilities are generally not homothetic or
homogeneous.

\paragraph{Fisher}
The utility is not homogeneous, so it is not compatible
with the Fisher convex formulation.

\paragraph{Exchange}
We find that 
\begin{align*}
\log(\nabla u(x)^T x) - \log(\nabla_j u(x)) =
\log\left(\sum_{k=1}^n a_k - \frac{a_k b_k}{x_k+b_k} \right) - \log a_j + \log (x_j + b_j),
\end{align*}
which is indeed concave.

\paragraph{Demand}
From the optimality conditions, we find that
\[
x_j = \max(0, \tau a_j/p_j - c_j),
\]
where $\tau \geq 0$ is some constant such that $\sum_j p_j x_j = p^Tb$.
This is basically \emph{water-filling}.

\paragraph{Cutting plane}
warp, wgs, monotone?


\section{Computing equilibria via splitting}
introduce prox operators \cite{parikh2013proximal, boyd2011distributed}.

We can compute solutions to both the Fisher and exchange convex
formulations, problems (\ref{p-fisher}) and (\ref{p-exchange}), using
the alternating direction method of multipliers (ADMM) \cite{boyd2011distributed}.

We'll see that we can put both problems into the form

\begin{equation}
\label{p-admm}
\begin{array}{ll}
\mbox{minimize} & \sum_{i=1}^m f_i(x_i, \phi_i) \\
\mbox{subject to} & \sum_{i=1}^m x_i \leq B\\
& \phi_i = \Phi, \quad \forall i
\end{array}
\end{equation}
for appropriately chosen functions $f_i$.
Recall that the vector $B \in \reals^n_{++}$ gives the total
amount of goods available in the market. In the Fisher problem,
$B$ is given.
In the exchange problem, $B = \sum_i b_i$.
The variables $x_i$ give the local allocations for agent $i$, and
$\phi_i \in \reals^n_{++}$
are the local log-prices, which must agree globally through the
consensus variable $\Phi$.
(In the Fisher formulation, $f_i$ will not depend on $\phi$.)

We describe the $f_i$ functions in the next two subsections, and provide an
ADMM implementation to solve the problems in the third section.

\subsection{Fisher}
We can simply assign
\[
f_i(x_i, \phi) = -w_i \log u_i(x_i) + I_{\lbrace x \mid x \geq 0 \rbrace}(x_i),
\]
where the second term is the indicator function for the positive orthant.
We see that problem~(\ref{p-admm}) is equivalent to
problem~(\ref{p-fisher}).

\subsection{Exchange}

For agent $i$, let $f_i(x_i, \phi)$ be the indicator function for the
constraints
\[
\begin{array}{c}
\log(\nabla u_i(x_i)^T x_i) - \log(\nabla_j u_i(x_i)) + \phi_j \geq  \log\left(\sum_k b_{ik} e^{\phi_{k}}\right),\quad \forall j\\
x_i \geq 0.
\end{array}
\]
We see that problem~(\ref{p-admm}) is equivalent to problem~(\ref{p-exchange}).

\subsection{XXX: note}
the admm algorithm in the next section is written for the more general case of 
the exchange problem, so it includes $\phi$ variables.
The fisher problem doesn't involve $\phi$, so it may not really make sense
to write a single algorithm for both settings. figure out later...

\subsection{ADMM algorithm}
We can use an ADMM-based splitting method \cite{boyd2011distributed} to
solve the market equilibrium problem written in the form of
problem~(\ref{p-admm}).
We will be interested in exploiting sparsity in the market, such
as each agent only being interested buying and selling a small subset 
of all the possible goods.
For this purpose, we will use an indexing notation in this section which is
more suitable to represent this sparsity and will simplify our resulting
ADMM equations.

Let $\mathcal{G}$ be an indexing set for all goods
in the market.
Agent $i$ will purchase amount $(x_i)_g$ of good $g \in \mathcal{G}$.
(We will also write $x_{ig}$ to lighten notation.)
Agent $i$'s utility is a function of the bundle $x_i \in \reals_{+}^{|G_i|}$,
where $G_i \subset \mathcal{G}$ is the set of goods involved in the utility
function $u_i$.

Note that we don't specify the actual ordering of the elements
of the vector $x_i$, as we will just care about the values associated with
each good.
When we do any operations (such as addition or scalar product) with
two vectors corresponding to the same subset of goods, we will
assume that the operation is done in a way that respects the local ordering of
the goods.

Agent $i$ is also initially endowed with a set of goods
$H_i \subset \mathcal{G}$,
with allocation values $(b_i)_g = b_{ig}$, where $b_i \in \reals_{++}^{|H_i|}$.
Again, $B \in \reals^n_{++}$ will be the total amount of goods in the market, which, in our
current notation, we can represent as
\[
B_g = \sum\limits_{i \in H^{-1}_g} b_{ig}
\]
for each good $g \in \mathcal{G}$.

We will treat $G$ and $H$ as \emph{relations},
using subscript notation instead of function notation.
As we have seen, $G_i$ corresponds to the set of goods that agent $i$ is
interested
in possibly purchasing.
The relation notation will also allow us to
write $G^{-1}_g$ to denote the set of agents which are interested in
purchasing good $g$.
Similarly, $H^{-1}_g$ is the set of agents initially endowed with some
positive amount of good $g$.

In the ADMM algorithm, each agent will have a local opinion for the prices
of the goods he is interested in purchasing or selling.
We represent his local log-prices
by $\phi_i \in \reals_+^{|G_i \cup H_i|}$.

Note that
\[
\bigcup_{i=1}^n G_i = \bigcup_{i=1}^n H_i = \mathcal{G}.
\]

For any variable $z \in \reals^{|\mathcal{G}|}$ which contains values for
each good in the market, we may write $z_{G_i}$ to represent the subvector
of $z$ whose elements correspond to the goods in $G_i$.

For agent $i$, let $f_i(x_i, \phi_i)$ be the indicator function for the
constraints
\[
\begin{array}{c}
\log(\nabla u_i(x_i)^T x_i) - \log(\nabla_g u_i(x_i)) + \phi_{ig} \geq  \log\left(\sum\limits_{k \in H_i} b_{ik} e^{\phi_{ik}}\right),\quad \forall g \in G_i\\
x_{ig} \geq 0, \quad \forall g \in G_i.
\end{array}
\]

\paragraph{ADMM problem form} Then the problem we'd like to solve is given by
problem (\ref{p-admm}), but we'll rewrite it here using this section's notation.
We will also replace the inequality constraint on purchased goods with an
equality, since this simplifies the ADMM iteration, and we know that
the constraint is tight at equilibrium.
The reformulated problem is
\[
\begin{array}{ll}
\mbox{minimize} & \sum_i f_i(x_i, \phi_i) \\
\mbox{subject to} & \sum\limits_{i \in G^{-1}_g} x_{ig} = B_g,\quad \forall g \in \mathcal{G}\\
& \phi_{ig} = \Phi_g,\quad \forall i \in G^{-1}_g \cup H^{-1}_g,\ \forall g \in \mathcal{G}.
\end{array}
\]

The first set of constraints are the global resource constraints; the amount of
goods purchased by agents cannot exceed the total initial endowed amount.

The second set of constraints are consensus constraints on the price variables,
$\phi_i$.
Each agent will have an opinion on the price of a good if he is buying or selling it.
The constraint says that all agents with an opinion, $\phi_{ig}$, on the price
of good $g$ must agree.
They agree through a global consensus variable $\Phi$.


\paragraph{ADMM algorithm}
Note that we have a generalized consensus problem in the price variables,
$\phi$, and a generalized sharing problem in the allocation variables, $x$.

The resulting ADMM algorithm is 
\begin{align}
\label{a-xtild}
\tilde{x}^k_g &:= \frac{1}{|G^{-1}_g|} \left( \sum_{i \in G^{-1}_g} x^k_{ig} - \sum_{i \in H^{-1}_g} b_{ig}\right)\\
\label{a-phibar}
\bar{\phi}^k_g &:= \frac{1}{ |G^{-1}_g \cup H^{-1}_g| } \sum_{i \in G^{-1}_g \cup H^{-1}_g}\phi^k_{ig}\\
u^{k+1} &:= u^k + \tilde{x}^k\\
w_i^{k+1} &:= w_i^k + \phi^k_i - \bar{\phi}^k_{G_i}\\
\label{a-prox}
x_i^{k+1}, \phi_i^{k+1} &:= \mbox{prox}_{f_i}(x_i^k - \tilde{x}^k_{G_i} - u^{k+1}_{G_i},
\bar{\phi}^k_{G_i} - w_i^{k+1}).
\end{align}

This algorithm needs to be explained.
In equation~(\ref{a-xtild}), $\tilde{x}^k_g$ gives a measure of the violation of
the global resource constraint, normalized by the number of agents participating
in considering that good.
In equation~(\ref{a-phibar}), $\bar{\phi}^k_g$ gives the average of the price
opinions $\phi^k_g$ over all agents with an interest in the price of good $g$.
Dual variables $u^k$ and $w^k_i$ are updated in the next two equations.
Note that $\tilde{x}^k$, $\bar{\phi}^k$, and $u^k$ are \emph{global} variables
and that $w_i^k$, $x_i^{k+1}$, and $\phi_i^{k+1}$ are \emph{local} variables
for each agent. In equation~(\ref{a-prox}), each agent computes the projection
onto his local optimality constraints.
The input to this proximal operator is formed by combining
local and global variables.
For example, since $\tilde{x}^k$ is a global variable with values for each good in the market, we use the notation $\tilde{x}^k_{G_i}$ to denote the subset of this vector corresponding to goods in $G_i$.
Each agent emits the results of their proximal operator and the
global computation of $\tilde{x}^{k+1}_g$ and $\bar{\phi}^{k+1}_g$ continues in the next step.

The algorithm above is decentralized and parallelizable.
Each agent can compute their prox operator separately and in parallel.
The results of these prox operators are aggregated, averaged, and then distributed
back to the individual agents for the next iteration.



\section{Other solution methods}

\subsection{Infinite LP/cutting plane methods}
That WGS gives a cutting plane is proved in \cite{arrow1959stability}.
(supposedly... check, as the papers have similar names.)

\subsection{Tatonnement}

\subsection{Weighted total utility}
put in the fisher stuff here. arrow-debreu can be solved this way, but we
don't know the weights. introduce re-weighting idea.

\subsection{CVXPY and SCS}

\subsection{Monotone operators}
For many utility functions,
%todo when?
the excess demand function
(or correspondence,
or relation) of an agent is a
monotone operator.
That is,
\begin{equation}
\label{e-monotone}
\left[z(p) - z(q) \right]^T \left[p - q\right] \leq 0.
\end{equation}

If we assume that Walras' Law holds, \ie, each agent spends their entire
budget, or $p^T z(p) = p^T x(p) - p^T b = 0$, then the operator monotonicity has
a nice economic interpretation in terms of \emph{preferences}.

Expanding inequality~(\ref{e-monotone}) and applying Walras' Law, we find that
\[
p^T z(q) + q^T z(p) \geq 0.
\]

We thus see that the implication
\begin{equation}
\label{e-preference}
p^T z(q) \leq 0 \implies q^T z(p) \geq 0
\end{equation}
is equivalent
%todo or just implies? is it actually true that they are equivalent?
to inequality~(\ref{e-monotone}) whenever Walras' Law holds.

We interpret (\ref{e-preference}) as revealing that the agent prefers prices
$p$ over prices $q$. If $p^T z(q) \leq 0$, then $p^T x(q) \leq p^T b$, implying
that both bundles $x(p)$ and $x(q)$ are affordable (feasible) at prices $p$.
Since $x(p)$ is an optimal bundle at prices $p$, we know that it must be at least
as preferred at $x(q)$, or that $u(x(p)) \geq u(x(q))$.

Inequality~(\ref{e-monotone}) already shows that implication~(\ref{e-preference})
holds, but to interpret the result economically, suppose that the
implication did not hold, \ie, that $q^T z(p) < 0$. We would then have that
$q^T x(p) < q^T b$, which implies that $x(p)$ and $x(q)$ are both affordable
bundles at at prices $q$. However, we know that $u(x(p)) \geq u(x(q))$, so
bundle $x(p)$ must be optimal, and Walras' Law implies $q^T x(p) = q^T b$, which
gives a contradiction. 

In the case of strict inequality, if $p^T z(q) < 0$, then $q^T z(p) > 0$.
That is, $x(p)$ is revealed as preferred to $x(q)$, and $x(p)$ is not within
budget at prices $q$, so prices $p$ are preferred by the agent.

Note that if each agent's excess demand function is a monotone operator, then
the market's aggregate excess demand function is also monotone. If these
operators are \emph{maximal} monotone, then we can use the this framework for
solving these problems.
%todo However, it is unclear if and when these operators are maximal.

Even in the case that the operators are monotone but not maximal, we still get
a cutting-plane by computing the demand function. That is, for any prices $p$
and optimal prices $p^*$, we have that
\[
p^{*T} z(p) \geq 0,
\]
which gives us a cutting-plane in the price simplex.
A cutting-plane algorithm can then be used to compute the optimal prices

\paragraph{Computing the resolvent}
The resolvent of the excess demand operator is
\[
(I + \lambda z)^{-1}(p) = q,
\]
which we parse to interpret as
\[
p = q + \lambda x(q) - \lambda b.
\]
That is, given prices $p$, we must find prices $q$ such that the agent's UMP
at prices $q$ satisfy the above quality.

XXX: We can write down the optimality conditions, but these may put the same
constraints on the utility functions as the Ye formulation. We can always
substitute $p = q + \lambda x(q)$ into the opt conditions to remove either
$q$ or $x$. But we end up with a non-convex problem. We optimize over the a
non-convex region, something like the space outside of a unit ball.
We may also be able to formulate the problem as being bi-convex in $x$ and $q$,
but I'm not sure how helpful that would be.

\subsection{Cutting plane}
\paragraph{Approximations?}
can i approximate the exponential cone to do an approximate prox evaluation
at each step, allowing for SOCP solvers that can't handle exponential cones?

what if we approximate everything with a quadratic around the current iterate
for an approximate projection? subset or superset of actual agent cutting set?
does this simplification make updates with an analytic form, allowing for very
fast iteration?


\section{Computational examples}
%\missingfigure{Add a yet to be made figure here.}

\section{Application examples}
\cite{shoven1992applying}

see \cite{shoven1992applying} for examples and applications of nested CES functions

\begin{itemize}
\item traffic
\item routing
\item spectrum management
\item others?
\end{itemize}

\appendix
\section{Cutting planes from gross substitutability}
Prove that when our utility functions satisfy gross substitutability,
we are able to obtain cutting
planes in the price simplex by evaluating the aggregate demand function.

\section{Cutting planes from the weak axiom of revealed preferences}
Prove that when our utility functions satisfy the
weak axiom of revealed preferences, we are able to obtain cutting
planes in the price simplex by evaluating the aggregate demand function.

\section{Convex formulation for exchange}
We prove that solutions to the convex formulation %todo ref
do provide equilibrium prices and allocations.

\section{Convex formulation for Fisher}
\label{sec:fisher_proof}
We prove that solutions to the convex formulation %todo ref
are indeed equilibrium allocations.

\newpage
\bibliographystyle{alpha}
\bibliography{bibliography}

\end{document}


