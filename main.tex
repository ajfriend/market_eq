\documentclass{article}
\usepackage[utf8]{inputenc}

\title{Computing market equilibria via convex optimization}
\author{AJ Friend \and Stephen Boyd}
\date{\today}

\begin{document}

\maketitle

\section{Introduction}

\section{Market definitions}
\subsection{Arrow-Debreu}

\subsection{Fisher}

\section{Convex formulation, equilibrium, and optimality conditions}
\subsection{Fisher}
\subsection{Arrow-Debreu}
\subsection{Alternative formulations}
any benefit to these? XXX: cite

\section{Utility functions}
\subsection{Definitions}
homothetic, homogeneous, log-concave


\subsection{Arrow-Debreu}
what work, what are representable
\begin{itemize}
\item complicated CES functions, nested, etc..
\item rational powers represented through SOCP
\end{itemize}
\subsection{Fisher}
can we transform homothetic to homogeneous of degree 1? or is that transformation completely non-constructive?

\subsection{Models with production?}
should we include?

\section{Solution methods}

\subsection{CVXPY and SCS}
\subsection{Exploiting sparsity}
\subsection{Splitting}
\subsection{Cutting plane}
\subsection{Approximations?}
can i approximate the exponential cone to do an approximate prox evaluation
at each step, allowing for SOCP solvers that can't handle exponential cones?

\section{Huge, distributed models}
\end{document}






























