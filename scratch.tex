\documentclass{article}
\usepackage[utf8]{inputenc}
\usepackage{todonotes}
\usepackage{amssymb,amsmath}


\input defs.tex

\title{notes on duality}
\author{AJ Friend}
\date{\today}

\begin{document}

\maketitle

\section{Convex conjugate}
Consult \cite{boyd2009convex} for background on duality.
The definition of the conjugate function is
\[
f^*(y) = \sup_{x \in \dom f} \left( y^T x - f(x) \right).
\]

\paragraph{Alternative representation}
Note that if we see the conjugate as the optimal value of an optimization
problem, then we get the equivalence
\[
y = \nabla f(x) \iff f^*(y) = y^T x - f(x) = \nabla f(x)^T x - f(x).
\]
This equivalence provides a duality.
We can
\begin{itemize}
\item generate (uncontrolled) points of the $f^*$ by plugging (controlled) $x$
values into $\nabla f$, and
\item determine $f^*(y)$ by solving the gradient equation $y = \nabla f(x)$.
\end{itemize}
\todo[inline]{If $f^{**} = f$, then we can go both ways.}

\paragraph{Fenchel’s inequality} The inequality follows from the definition:
\[
f(x) + f^*(y) \geq x^T y
\]
Note that equality occurs when $y = \nabla f(x)$.

\section{Duality}
A primal problem has the form
\[
\begin{array}{ll}
\mbox{minimize} & f(x)\\
\mbox{subject to} & Ax = b\\
& x \geq 0.
\end{array}
\]

The dual problem has the form
\[
\begin{array}{ll}
\mbox{maximize} & -f^*(\tau - A^T \lambda) - b^T \lambda \\
\mbox{subject to} & \tau \geq 0
\end{array}
\]

\paragraph{Optimality conditions}
The optimality conditions are given by
\begin{align*}
A x &= b \\
x &\geq 0 \\
\tau &\geq 0 \\
f(x) + f^*(\tau - A^T \lambda) + b^T \lambda &\leq 0,
\end{align*}
where the first three conditions are primal and dual feasibility,
and the last condition is an upper bound on the duality gap.
(Since the duality gap is always nonnegative for a feasible pair,
this last condition implies zero duality gap.)
Implicitly, we assume the same constraints as on the domains of $f$ and $f^*$.

\paragraph{Zero duality gap}
To see the zero duality gap condition algebraically, recall Fenchel's inequality.
We have that
\[
f(x) + f^*(\tau - A^T \lambda) + b^T\lambda \geq x^T (\tau - A^T \lambda) + b^T\lambda
= x^T \tau \geq 0,
\]
which implies equality, and, in particular, the complementarity condition
$x^T \tau = 0$.
The equality triggers the equality condition in the Fenchel inequality, which
means that $\tau - A^T \lambda = \nabla f(x)$.
These two conditions give us
exactly the KKT conditions for the problem.
The next section shows this result in another (but essentially the same) way.

\paragraph{Equivalence with KKT conditions}
The optimality conditions given above are equivalent to the standard KKT
conditions. To see this, note that since, via the above constraints,
$\tau - A^T \lambda \in \dom f^*$, so there is some $x$ such that

\begin{align*}
\nabla f(x) &= \tau - A^T \lambda \\
f^*(\tau - A^T \lambda) &= \left( \tau - A^T \lambda \right)^T x - f(x).
\end{align*}

Adding the first condition and substituting the second into the optimality
conditions above gives us the usual KKT conditions:

\begin{align*}
A x &= b \\
x &\geq 0 \\
\tau &\geq 0 \\
\tau - A^T \lambda &= \nabla f(x)\\
\tau^T x &= 0.
\end{align*}

\todo[inline]{Now, we'll show the other direction of the equivalence.}
\todo[inline]{For LP, different forms gives complementarity and zero duality
gap versions of KKT.}
\todo[inline]{The duality of complementarity and duality gap must be baked into
the duality derivation.}
\todo[inline]{Can I make the first version into a convex program by using the
Fenchel inequality to show that the last sum is bounded below by zero?
Of course I can! If not fenchel, then certainly because we are bounding the
the duality gap, so we only want it less than zero!}
\todo[inline]{Can I use this to write Ye's optimality conditions without
gradients of functions?}
\paragraph{Example}
For the function $f(x) = c^T x$, we have the conjugate
$f^*(y) = 0$ if $y=c$ and $f^*(y) = +\infty$ otherwise.
This allows us to recover the usual LP KKT conditions:
\begin{align*}
A x &= b \\
x &\geq 0 \\
\tau &\geq 0 \\
c^T x + b^T \lambda &= 0\\
\tau - A^T \lambda &= c.
\end{align*}
Note that the fourth condition, that the duality gap is zero, can be
re-written as a complementarity condition, since

\begin{align*}
& c^T x + b^T \lambda = 0 \\
\implies& (\tau - A^T \lambda)^T x + b^T \lambda = 0 \\
\implies& \tau^Tx - \lambda^T A x + b^T \lambda = 0 \\
\implies& \tau^Tx - \lambda^T b + b^T \lambda = 0 \\
\implies& \tau^Tx  = 0.
\end{align*}

\section{Fisher opt conditions}
The primal Fisher problem has the form
\[
\begin{array}{ll}
\mbox{minimize} & - u_i(x_i)\\
\mbox{subject to} & p^T x_i \leq w_i\\
& x_i \geq 0.
\end{array}
\]

The dual Fisher problem has the form
\[
\begin{array}{ll}
\mbox{maximize} & -(-u_i)^*(\tau - p y_i) - w_i y_i\\
\mbox{subject to} & \tau_i \geq 0\\
& y_i \geq 0.
\end{array}
\]
\todo[inline]{write down KKT conditions. check with example of linear utility}

\section{Arrow-Debreu conditions}
The primal Arrow-Debreu problem has the form
\[
\begin{array}{ll}
\mbox{minimize} & - u_i(x_i)\\
\mbox{subject to} & p^T x_i \leq p^T b_i\\
& x_i \geq 0.
\end{array}
\]

The dual problem has the form
\[
\begin{array}{ll}
\mbox{maximize} & -(- u_i)^*(\tau_i - p y_i) - p^T b_i y_i\\
\mbox{subject to} & y_i \geq 0\\
& \tau_i \geq 0.
\end{array}
\]

We can also write down the optimality conditions for the equilibrium allocation,
which includes a global resource constraint.

\begin{align*}
-\nabla u_i(x_i) + p y_i &= \tau_i\\
y_i p^T x_i &= y_i p^T b_i \\
\tau_i^T x_i &= 0\\
p^T x_i &\leq p^T b_i\\
x_i, y_i, \tau_i &\geq 0\\
\sum_{i} x_{ij} &\leq \sum_{i} b_{ij}
\end{align*}

First, we'll simplify the constraints by removing $\tau$:

\begin{align*}
p y_i &\geq \nabla u_i(x_i) \\
y_i p^T x_i &= y_i p^T b_i \\
y_i p^T x_i &= \nabla u_i(x_i)^T x_i\\
p^T x_i &\leq p^T b_i\\
x_i, y_i &\geq 0\\
\sum_{i} x_{ij} &\leq \sum_{i} b_{ij}
\end{align*}

Next, we remove the $y$ variables by combining the second and third constraint
to get that
\[
y_i = \frac{\nabla u_i(x_i)^T x_i}{p^T b_i},
\]
and plug it into the first constraint to get the new set of constraints:
\begin{align*}
\frac{\nabla u_i(x_i)^T x_i}{p^T b_i} p &\geq \nabla u_i(x_i) \\
p^T x_i &\leq p^T b_i\\
x_i, y_i &\geq 0\\
\sum_{i} x_{ij} &\leq \sum_{i} b_{ij}.
\end{align*}

We'll drop the second constraint, as it is implied by the others:
\begin{align*}
\nabla u_i(x_i)^T x_i p_j &\geq \nabla_j u_i(x_i) p^T b_i\quad \forall i,j \\
\sum_{i} x_{ij} &\leq \sum_{i} b_{ij}\quad \forall i\\
p, x_i &\geq 0\quad \forall i.
\end{align*}

\paragraph{Equivalence with original constraints}
To see the equivalence with the original constraints, take the first inequality,
multiply by $x_{ij}$ and sum over $j$ to get
\begin{align*}
\sum_j \nabla u_i(x_i)^T x_i p_j x_{ij} &\geq \sum_j \nabla_j u_i(x_i) x_{ij} p^T b_i \\
\implies \nabla u_i(x_i)^T x_i p^T x_i &\geq \nabla u_i(x_i)^T x_i p^T b_i \\
\implies p^T x_i &\geq p^T b_i.
\end{align*}

Note that the last constraint implies that
\[
p_j \sum_i x_{ij} \leq p_j \sum_i b_{ij}.
\]

Now, we have that

\begin{align*}
\sum_i p^T b_i &\leq \sum_i p^T x_i\quad \text{(by something)}\\
&= \sum_j p_j \sum_i x_{ij} \\
&\leq \sum_j p_j \sum_i b_{ij}\quad \text{(by something)}\\
&= \sum_i p^T b_i.
\end{align*}

This implies equality holds throughout, so, in particular,
\begin{align*}
p^T x_i &= p^T b_i\\
\sum_i x_{ij} &= \sum_i b_{ij}
\end{align*}

This recovers all the original constraints.

\section{Arrow-Debreu conditions with conjugate functions}
The dual of an agent's maximization problem in the Arrow-Debreu setting
is given above, but involves a conjugate function.
We see that the full KKT conditions for equilibrium are given by
primal and dual feasibility and zero duality gap:

\begin{align*}
-u_i(x_i) + (-u_i)^*(\tau_i - p y_i) + p^T b_i y_i &\leq 0\\
p^T x_i &\leq p^T b_i \\
\sum_i x_{ij} &\leq \sum_i b_{ij}\\
p, x_i, y_i, \tau_i &\geq 0.
\end{align*}
Of course, we can also obtain the usual form for the KKT conditions.
Note that Fenchel's inequality implies
\[
-u_i(x_i) + (-u_i)^*(\tau_i - p y_i) + p^T b_i y_i
\geq x_i^T \tau_i - x_i^T p y_i + p^T b_i y_i,
\]
which implies that
\begin{align*}
x_i^T \tau_i &= 0\\
x_i^T p y_i &= p^T b_i y_i,
\end{align*}
which are the KKT complementarity conditions.
The gradient condition follows from the equality case of the Fenchel inequality,
which gives that
\[
\tau_i - p y_i = - \nabla u_i(x_i).
\]
This gives us the complete set of KKT conditions.

Unfortunately, the conjugate form of the KKT conditions doesn't seem to lend
itself to any useful transformations which might help us to attack the problem
in any way other than Ye's formulation.

\newpage
\bibliographystyle{alpha}
\bibliography{bibliography.bib}


\end{document}